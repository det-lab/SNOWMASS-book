
%  macros specifically needed for the Glossary 

\def\dzero{ \not{\hbox{\kern-4pt \mbox{O}}} }

%%%%%%%%%%%%%%%%%%%%%%%%%%%%%%%%%%%%%%%%%%%%%%%%%

\setcounter{chapter}{12}

\chapter{Glossary}
\label{chap:gloss}

\bigskip
\bigskip

{\bf This glossary, from the 2013 volume, is a start on the glossary
required for the 2021 volume.    Please send requests for changes or
new entries to Michael Peskin (mpeskin@slac.stanford.edu), who is
maintaining the production version.}

\def\glossterm#1{\noindent {\bf #1\ : } }

\bigskip
\bigskip


%%%%%%%%%%%%%%%%%%%%%%%%%%%%%%%%%%%

\glossterm{\boldmath$0\nu\beta\beta$}  Neutrinoless double-beta decay, a
nuclear decay process in which two electrons and zero neutrinos are 
emitted.

\glossterm{\boldmath$2\nu\beta\beta$}  Double-beta decay, a
nuclear decay process in which two electrons and two neutrinos are 
emitted.

\glossterm{2HDM}  Two-Higgs-Doublet Model, a model with two Higgs
 fields, usually with natural flavor conservation in Higgs boson couplings.

\glossterm{AAAS}  American Association for the Advancement of Science.

\glossterm{ACTA}  Augmented Cherenkov Telescope Array, the next-generation
atmospheric Cherenkov telescope project for gamma ray detection.

\glossterm{ADMX}  Axion Dark Matter Experiment, an experiment at the
University of Washington searching for axions using a large
electromagnetic cavity in a static magnetic field.

\glossterm{ADS}  Accelerator-Driven Systems, a technology for driving
nuclear reactors with particle beams.

\glossterm{AGILE}  Astro-Rivelatore Gamma a Immagini Leggero, a space-based
X-ray and gamma ray observatory.

\glossterm{AGN}  Active Galactic Nucleus.

\glossterm{AIP} American Institute of Physics.

\glossterm{ALCF} Argonne Leadership Computer Facility.

\glossterm{ALICE}  A Large Ion Collider Experiment, an experiment
at the {\bf LHC} focusing on heavy ion collisions and the quark-gluon
plasma.

\glossterm{ALP}  Axion-Like Particle.

\glossterm{ALPS, ALPS-II}  Any Light Particle Search, a series of
experiments at {\bf
  DESY} searching  for 
{\bf axions} and similar particles using laser light conversion in a strong
 magnetic field.

\glossterm{AMANDA}  Antarctic Muon and Neutrino Detector Array, the
first-generation neutrino telescope experiment in Antarctica.


\glossterm{AMR} Adaptive Mesh Refinement.

\glossterm{AMS, AMS-02}  Alpha Magnetic Spectrometer, an antimatter detector located on the
International Space Station.

\glossterm{ANITA}  Antarctic Transient Antenna,  a balloon experiment
in Antarctica for detection of radio signals of ultra-high-energy
neutrino events. 

\glossterm{ANL} Argonne National Laboratory.

\glossterm{ANTARES}  Astronomy with a Neutrino Telescope and Abyss
environmental RESearch,  a neutrino telescope experiment in the
Mediterranean Sea.

\glossterm{APEX}  A Prime EXperiment, an experiment at {\bf JLab}
  searching for the production of dark photons using an intense
 electron beam.

\glossterm{APS}  American Physical Society.

\glossterm{ARA}  Askaryan Radio Array, a  radiofrequency neutrino
 antenna experiment at the South Pole.

\glossterm{ArgoNeuT} Argon Neutrino Teststand, a small liquid argon
{\bf TPC} experiment at {\bf Fermilab}.

\glossterm{ARIANNA}  Antarctic Ross Ice-shelf ANtenna Neutrino Array,
a radiofrequency neutrino antenna experiment in Antarctica.

\glossterm{ART}  An Intensity Frontier software framework developed by the 
   {\bf Fermilab} Scientific Computing Division and used by {\bf
     NOvA}, {\bf mu2e}, {\bf Muon \boldmath$g-2$}, 
   {\bf LArSoft} , {\bf DarkSide}, {\bf LArIAT}, and others.

\glossterm{ASIC}  Application Specific Integrated Circuit, a silicon
chip processor designed for a particular purpose.

\glossterm{ASTA}  Advanced Superconducting Test Accelerator, an 
      accelerator test facility at {\bf Fermilab}.

\glossterm{ATCA}  Advanced Telecommunications Computing Architechture.

\glossterm{ATF, ATF2}  Accelerator Test Facility, an accelerator test
complex
 at {\bf KEK} built to study beam dynamics issues for the {\it ILC}. 

\glossterm{ATLAS}  A Toroidal LHC ApparatuS, a large experiment at the 
{\bf LHC} optimized for high transverse momentum 
particle production.

\glossterm{ATR} Advanced Test Reactor, a research reactor at the Idaho National Laboratory.

\glossterm{Auger}  Pierre Auger Observatory, experiment in Argentina that detects
ultra-high-energy cosmic rays through both fluorescence and surface
water Cherekov detectors.

\glossterm{Axion} A hypothetical, very light particle, with
couplings to quarks that reverse under {\bf P} transformations.  Such
particles could make up the dark matter of the Universe.

\glossterm{BaBar}  An experiment at {\bf SLAC}, using $\ee$ annihilation at
approximately 10~GeV to study rare heavy quark and lepton decays. 

\glossterm{Baikal} A neutrino telescope in Lake Baikal in Siberia.

\glossterm{Baksan} An underground laboratory in the Caucasus mountains in
Russia.

\glossterm{BAO}  Baryon Acoustic Oscillations.

 \glossterm{Baseline}  In neutrino physics, the distance between the
 neutrino production point, near an accelerator, and the neutrino
 detector.

\glossterm{BELLA} Berkeley Lab Laser Accelerator, a facility at {\bf
  LBNL} for  the development of laser-driven accelerators.

\glossterm{Belle}  An experiment at {\bf KEK}, using $\ee$ annihilation at
approximately 10~GeV to study rare heavy quark and lepton decays. 

\glossterm{Belle-II}  An experiment at {\bf KEK}, using $\ee$ annihilation at
approximately 10~GeV to study rare heavy quark and lepton decays with 
50-100 times the  data set produced by {\bf BaBar} and {\bf Belle}. 

\glossterm{BESIII}  An experiment at {\bf IHEP}, using $\ee$ annihilation at
approximately 3--4~GeV to study charm quark and tau lepton physics. 

\glossterm{BEST} Baksan Experiment on Sterile Transitions, a proposed 
radioactive source experiment in Russia with a gallium detector.

\glossterm{BINP} Budker Institute for Nuclear Physics, a high-energy
physics laboratory in Novosibirsk, Russia.

\glossterm{BNB} Booster Neutrino Beam , a  neutrino beamline at
{\bf Fermilab} using the Booster.

\glossterm{BNL} Brookhaven National Laboratory.

\glossterm{Borexino}  A solar neutrino experiment at {\bf Gran
  Sasso}. 

\glossterm{Boson} A type of elementary particle whose intrinsic spin is an
integer (0, 1, 2, $\ldots$) multiple of  $\hbar$.  Such particles can make
up a macroscopic force field such as the electromagnetic field.

\glossterm{BOSS} Baryon Oscillation Spectroscopic Survey, a galaxy
survey aiming to measure baryon acoustic oscillations.

\glossterm{BR} Branching Ratio, the probability that an unstable particle
decays to a particular final state.

\glossterm{BSM}  Beyond the Standard Model, a reference to new physics 
associated with an extension of the {\bf SM}. 

\glossterm{C} Charge conjugation, the interchange of particles and 
antiparticles.


\glossterm{CALICE} CALorimeter for Linear Collider Experiment, an
experimental collaboration aimed at improving the technology of 
hadron calorimeters, especially with the {\bf PFA} method.

\glossterm{Canfranc Underground Laboratory}  An underground scientific facility
in a former railway tunnel in the Spanish Pyrenees under Monte Tobazo in 
Canfranc, Spain.

\glossterm{Capability}  In computing, refers to high-speed, highly
parallel
    computing that might require  a large fraction of a supercomputer.

\glossterm{Capacity}  In computing, refers to computing in which 
      many moderately parallel jobs are run alongside one another.

\glossterm{CAPTAIN} Cryogenic Apparatus for Precision Tests 
of Argon INteractions, a liquid argon R\&D detector.

\glossterm{CAST} CERN Axion Solar Telescope, an experiment 
     at {\bf CERN} searching for {\bf axions} radiated from the Sun.

\glossterm{CC} Charged Current weak interactions.
 
\glossterm{CCD} Charge Coupled Device,  a class of pixel silicon detectors.

\glossterm{CDF}  Collider Detector at  Fermilab,  a large experiment at the 
{\bf Fermilab} Tevatron optimized for high transverse momentum 
particle production.

\glossterm{CDM}  Cold Dark Matter, a class of dark matter models in which 
the dark matter particles move at nonrelativistic speeds.


\glossterm{CDR} Conceptual Design Report.

\glossterm{Ce-LAND} A $^{144}$Ce source to be placed in {\bf KamLAND} to
study the reactor neutrino anomaly.

\glossterm{CENNS} Coherent Elastic Neutrino-Nucleus Scattering, an {\bf NC}
neutrino process and also a proposed experiment to be sited at the {\bf BNB}.

\glossterm{CE\&O}  Communication, Education, and Outreach, a topic 
of one of the ``frontiers'' in this study.

\glossterm{CERN}  Conseil Europ\'een pour la Recherche
Nucleaire, the major European high energy physics laboratory, located
in Geneva.

\glossterm{CESR} Cornell Electron Storage Ring, an $\ee$ colliding
beam accelerator at Cornell University that operated in the energy
range 3.5-12 GeV from 1979 to 2008.  The accelerator
is still in operation as a synchrotron light source (CHESS) and as an
accelerator physics testbed.

\glossterm{CesrTA}  CESR Test Accelerator, a configuration of  
{\bf CESR} to study the design of electron damping rings, in 
particular, for {\bf ILC}. 

\glossterm{Chameleon}  A new particle that has properties that 
depend on its environment.

\glossterm{CHIPS} CHerenkov detectors In mine PitS, a
 proposed experiment to use the {\bf Fermilab} beams and  
massive Cherenkov detectors in flooded mine pits.

\glossterm{CHOOZ} A first-generation reactor neutrino experiment
 located in Chooz, France.

\glossterm{CKM} Cabibbo-Kobayashi-Maskawa matrix, the matrix
  relating the weak interaction and mass eigenstates of quarks.

\glossterm{CL} (Statistical) Confidence Level.
 
\glossterm{CLEAN} Cryogenic Low Energy Astrophysics with Noble gases, 
a cryogenic noble liquid experiment for dark matter and solar neutrinos.

\glossterm{CLEO} A general-purpose particle detector at {\bf CESR}  studying heavy quark 
decays and spectroscopy.

\glossterm{CLFV} Charged Lepton Flavor Violation.

\glossterm{CLIC} Compact LInear Collider, a concept for an $\ee$ 
linear collider, with center of mass energies up to 3~TeV, based on 
two-beam acceleration.

\glossterm{CM} Center of Mass, the system for viewing a particle collision
or decay in which the overall system is at rest.

\glossterm{CMB} Cosmic Microwave Background, the approximately
isotropic microwave radiation in the universe created in the original
formation of atoms from electrons and ionized protons. 

\glossterm{CMS} Compact Muon Spectrometer, a large experiment at the 
{\bf LHC} optimized for high transverse momentum 
particle production.

\glossterm{CoGeNT}  COherent GErmanium Neutrino Technology, a germanium
detector for dark matter and other signals requiring low background.

\glossterm{COMET}  An experiment at {\bf J-PARC} searching for muon
 to electron conversion in the field of a heavy nucleus.

\glossterm{COUPP} Chicagoland Observatory for Underground Particle
Physics, a dark matter detector using a bubble chamber, now located
in the Sudbury Mine in Sudbury, Ontario.

\glossterm{CP} The combination of a {\bf C} and a {\bf P} transformation, 
converting particles to antiparticles, plus mirror (left-right) reflection.
This is a very accurate, but not perfect, approximate symmetry of nature.

\glossterm{CPAD} Coordinating Panel for Advanced Detectors, an
advisory panel on detector technology created by the {\bf APS DPF}.

\glossterm{CPT}  The combination of a {\bf C},  and a {\bf P}, and a 
{\bf T}  transformation, 
converting particles to antiparticles, plus mirror (left-right)
reflection, plus reversal of the direction of time.   Local quantum 
field theory predicts this to be a perfect symmetry of nature.

\glossterm{CPU}  Central Processing Unit of a computer.


\glossterm{CRADA} Cooperative Research and Development Agreement, an 
agreement between a government laboratory and a private company to 
pursue R\&D on a technology or project.

\glossterm{CSI} Coherent Scattering Investigations at the SNS, a
 proposed {\bf CENNS} search experiment for the {\bf SNS}.


\glossterm{CTA} Cherenkov Telescope Array, a planned large-area array of 
telescopes for high energy gamma rays.

\glossterm{CTF3}  CLIC Test Facilty 3, the most recent in a series of 
      test accelerators for {\bf CLIC} at {\bf CERN}. 

\glossterm{CUDA}  Compute Unified Device Architecture, which  defines a parallel computing 
  architecture for NVIDIA {\bf GPU}s.


\glossterm{CUORE} Cryogenic Underground Observatory for Rare Events, an
experiment searching for neutrinoless double  beta decay, located at
{\bf Gran Sasso}, Italy.

\glossterm{D\boldmath$\dzero$} D-zero, a large experiment at the {\bf Fermilab} Tevatron
optimized for high transverse momentum particle production.  

\glossterm{DAE\boldmath$\delta$ALUS} Decay At rest Experiment for $\delta_{CP}$ 
studies at the Laboratory for Underground Science, a neutrino oscillation
experiment based on beams created by cyclotrons.

\glossterm{DAMA}    DArk MAtter experiment, a scintillator-based dark 
 matter search experiment at {\bf Gran Sasso}.


\glossterm{DAMIC}  DArk Matter In CCD experiment, an experiment at 
      {\bf Fermilab} searching for light dark matter  particles.

\glossterm{DANSS} Detector of the reactor AntiNeutrino based on
Solid-state plastic Scintillator, a reactor neutrino experiment in
Russia.

\glossterm{Dark Light}  An experiment at {\bf JLab} searching for 
  dark photons using an {\bf FEL}.

\glossterm{DarkSide}  A dark matter search experiment at {\bf Gran
  Sasso}
       using a  liquid argon detector.

\glossterm{Daya Bay} A reactor neutrino experiment located near Daya Bay, China.

\glossterm{DeepCore} A low-energy extension to the {\bf IceCube} experiment
with a high density of photodetectors in a central region of the cube.


\glossterm{DES}  Dark Energy Survey.

\glossterm{DESI}  Dark Energy Spectroscopic Instrument.


\glossterm{DESY} Deutsches Elektronen SYnchrotron, the major
high energy physics laboratory in Germany, located in Hamburg.

\glossterm{DIRC} Detection of Internally Reflected Cherenkov light,
a detector using quartz bars for tracking and particle identification.
 
\glossterm{DIS} Deep Inelastic Scattering, a process of lepton scattering
from a nucleon or nucleus with large momentum transfer, especially when
only the lepton recoil is observed.

\glossterm{DM}  Dark Matter.

\glossterm{DOE} U.S. Department of Energy.

\glossterm{Double Chooz} A reactor neutrino experiment in Chooz,
France, utilizing detectors at two distances 
from the source.

\glossterm{DPF} Division of Particles and Fields, a division of the 
American Physical Society.

\glossterm{DREAM} Dual REAdout Method, a method for hadron 
calorimetry that corrects the charge/neutral response ratio by 
separate measurement of Cherenkov and scintillation light.


\glossterm{Drive beam}   A high energy particle beam used to 
 create an electromagnetic  field that can then accelerate another
beam to high energy.

\glossterm{DSNB} Diffuse Supernova Neutrino Background.


\glossterm{dSphs}  dwarf Spheroidal satellite galaxies of the Milky Way.


\glossterm{EAS}  Extensive Air Shower, produced by high energy cosmic rays in the
Earth's atmosphere.

\glossterm{EBL}  Extra-galactic Background Light.

\glossterm{ECHo} Electron Capture $^{163}$Ho experiment, 
a proposed neutrino mass microcalorimeter experiment.

\glossterm{EDM}  Electric Dipole Moment.

\glossterm{EFT} Effective Field Theory, a method of quantum field
theory in which the effects of new particles or interactions that
might be present at high energies are expressed by the addition of 
more complex operators to the equations of motion of the {\bf SM}.


\glossterm{ESA}  European Space Agency.

\glossterm{ESS} European Spallation Source, a future facility in
Lund, Sweden.

\glossterm{ESS\boldmath$\nu$SB} European Spallation Source Neutrino Super Beam, a
proposal to use the {\bf ESS}  proton linac to generate a neutrino
superbeam.

\glossterm{\boldmath$\ETmiss$}  Missing Transverse Energy, unobserved momentum
in  a  high-energy particle collision carried away by weakly interacting neutral particles.

\glossterm{Euclid}  A dark energy space mission currently under development by the
European Space Agency.

\glossterm{EVA} ExaVolt Antenna, a proposed balloon-based
 neutrino antenna experiment in Antarctica.

\glossterm{EW} ElectroWeak interaction, the unified description of 
the electromagnetic and weak interactions.

\glossterm{EWSB}  ElectroWeak Symmetry Breaking.

\glossterm{Exascale}  Of the order of $10^{18}$, used in computing to 
 refer to next-generation computation resources in memory or speed.

\glossterm{EXO} Enriched Xenon Observatory, an experiment searching
for neutrinoless double beta decay of the isotope Xe$^{136}$.

\glossterm{FACA}  Federal Advisory Committee Act, the legislation that 
governs questions of openness and transparency for certain Federal 
committees.  The act generally covers grant review panels of the {\bf NSF} but not 
those of the {\bf DOE}.

\glossterm{FACET}   Facility for advanced ACcelerator Experimental
Tests, a user facility at {\bf SLAC} for experiments on high-gradient 
electron accelerator technology.

\glossterm{FCNC} Flavor-Changing Neutral Current.

\glossterm{FEL}  Free Electron Laser, an electron accelerator that  produces
high-intensity
    coherent synchrotron radiation.

\glossterm{Fermi}  Femtometer =  $10^{-13}$cm.

\glossterm{Fermi-LAT}  Fermi Large Area Telescope, a space-based gamma-ray
detector.

\glossterm{Fermion} A type of elementary particle whose intrinsic spin is a
half-integer (${1\over 2}$, $ {3\over 2}$, $\ldots$) multiple of $\hbar$.  Such
particles can form rigid structures of matter such as atoms and atomic nuclei.

\glossterm{Fermilab}  Fermi National Accelerator Laboration, in
Batavia, Illinois. 

\glossterm{FFAG} Fixed-Field Alternating Gradient accelerator, an accelerator
design concept with time-independent magnetic bending fields and strong
focusing.

\glossterm{FPGA} Field Programmable Gate Array, a type of programmable 
integrated circuit.

\glossterm{FNAL} Fermi National Accelerator Laboratory, or {\bf
  Fermilab}.

\glossterm{FRIB}  Facility for Rare Isotope Beams, a linear
accelerator
  at Michigan State University for isotope production and nuclear
 structure research.


\glossterm{FrPNC}   Francium Parity Non-Conservation, at experiment
 at {\bf TRIUMF} to measure {\bf P}-violating atomic transitions in 
 francium
     atoms. 

\glossterm{FSR} Final State Radiation, radiation from a lepton, quark, or
gluon produced in a high-energy scattering process.

\glossterm{GALLEX} GALLium EXperiment, a radiochemical solar neutrino
experiment located at {\bf Gran Sasso}.

\glossterm{GammeV-CHASE}  An experiment at {\bf Fermilab} to search 
for {\bf axions}  using a laser beam in a strong magnetic field.

\glossterm{GAPS}  General Antiparticle Spectrometer, an experiment to detect
anti-matter, especially deuterons, produced by annihilating dark
matter particles.

\glossterm{GC}  Galactic Center.

\glossterm{GDE} Global Design Effort, the collaboration responsible for
the {\bf ILC} Technical Design Report.

\glossterm{GEANT4}  A library of simulation programs describing the 
passage of high-energy particles through matter.

\glossterm{GEM} Gas Electron Multiplier, a type of {\bf MPGD} ionization
detector.

\glossterm{GEMMA} Germanium Experiment for measurement of the Magnetic
Moment of Antineutrino,  a neutrino magnetic moment experiment at the
Kalinin nuclear power plant in Russia.

\glossterm{GERDA} Ge experiment searching for neutrinoless double
beta decay.

\glossterm{GeV} Giga-electron Volt ($10^9$ eV), the energy scale of the proton
mass and the subnuclear strong interactions.

\glossterm{GIM} Glashow-Iliopoulos-Maiani mechanism, a method for coupling
quarks to the weak interactions that avoid flavor-changing {\bf NC} processes,
realized in the {\bf SM}. 

\glossterm{GLACIER} Giant Liquid Argon Charge Imaging Experiment, a
proposed large liquid argon detector in Europe.

\glossterm{GNO} Gallium Neutrino Observatory, a radiochemical solar neutrino
experiment at {\bf Gran Sasso} (successor to {\bf GALLEX}).

\glossterm{GPU}   Graphics Processing Unit.

\glossterm{GR}  General Relativity, Einstein's theory of gravity.

\glossterm{Gran Sasso} An Italian national laboratory under a mountain of the
same name, about 120 km from Rome.

\glossterm{Grid}  In computing, a network of distributed computer and 
                    storage resources.   Typically, it refers to a network
                    in which users need not know at which node the
                    processor or data they are using resides.

\glossterm{GUT} Grand Unified Theory, a unified theory of all microscopic
particle interactions.

\glossterm{GZK neutrinos}  Greisen-Zatsepin-Kuzmin neutrinos, produced by
ultra-high energy cosmic ray protons scattering off {\bf CMB} photons.

\glossterm{HALO} Helium And Lead Observatory, lead-based supernova neutrino
detector at {\bf SNOLAB}.

\glossterm{HAWC}  High-Altitude Water Cherenkov, a gamma-ray detector currently
operating in Mexico.

\glossterm{HE-LHC}  High-Energy LHC, a proposed stage of the {\bf LHC} in which the
bending magnets are replaced by higher-field magnetics, to create $pp$
collisions
at center-of-mass energies of 26--33 TeV.

\glossterm{H.E.S.S.} High Energy Stereoscopic System, a telescope for
high-energy gamma rays seen as air showers in Cherenkov radiation, located
in Namibia.

\glossterm{HEP} High-Energy Physics, the generic term for the areas of
research described in this report.

\glossterm{HEPAP} High-Energy Physics Advisory Panel, a panel convened
by the U.S. {\bf DOE} and {\bf NSF} to advise the federal government
on high-energy physics research.

\glossterm{HERA} Hadron Elektron Ring Anlage, an electron-proton and
positron-proton collider at {\bf DESY} that operated from 1990 to 2007. 

\glossterm{HETDEX}  Hobby-Ebberly Telescope Dark Energy Experiment.
    

\glossterm{HFIR} High Flux Isotope Reactor, reactor facility at Oak Ridge
National Laboratory.

\glossterm{HHCAL} Homogeneous Hadron Calorimeter, a concept for hadron
calorimetry based on total absorption of hadrons in large crystals
equipped with dual readout (see {\bf DREAM}). 

\glossterm{HIGS} High Intensity Gamma-ray Source, a free electron
laser at Duke University.

\glossterm{HiRes}  High Resolution Fly's Eye cosmic ray experiment in Utah.

\glossterm{HLbL}  Hadronic Light-by-Light scattering contributions to
the
 muon anomalous magnetic moment.

\glossterm{HL-LHC} High Luminosity LHC, the highest-luminosity phase
of the {\bf LHC}
program, with plans to collect 3000~fb$^{-1}$ of data between 2022 and
2030. 

\glossterm{HPC} High Performance Computing, such as that  done using
supercomputers.

\glossterm{HPS}  Heavy Photon Search, an experiment at {\bf JLab}
searching for the decay of dark photons to $\ee$.

\glossterm{HSC}  Hyper-Suprime Cam, wide-field camera for the Subaru telescope.

\glossterm{HTC}  High Throughput Computing, i.e, data-intensive
computing.

\glossterm{HTS} High-Temperature Superconductivity.

\glossterm{Hyper-K} Hyper-Kamiokande, a proposed large water Cherenkov
detector at Kamioka in Japan. 

\glossterm{IACT}  imaging Atmospheric Cherenkov Telescope.

\glossterm{IAEA} International Atomic Energy Agency.

\glossterm{IAXO}  International Axion Observatory.

\glossterm{IBD} Inverse Beta Decay, usually referring to the reaction
$\bar{\nu}_e+p \rightarrow e^+ + n$.


\glossterm{ICAL} Iron CALorimeter atmospheric neutrino experiment at {\bf INO}.

\glossterm{ICARUS} Imaging Cosmic And Rare Underground Signals, a
liquid argon neutrino detector located at {\bf Gran Sasso}. 

\glossterm{IceCube} A neutrino telescope located at the
Amundsen--Scott South Pole station in Antarctica.

\glossterm{IceTray}  A software framework developed and used by the IceCube 
  experiment for both online and offline processing. 

\glossterm{IDS} International Design Study (for the Neutrino Factory).

\glossterm{IHEP} Institute of High Energy Physics of the Chinese Academy of
Sciences, the major particle physics laboratory in China, located in Beijing.

\glossterm{ILC} International Linear Collider, an electron-positron linear 
collider with design {\bf CM}  energy 500~GeV.

\glossterm{ILD} International Linear collider Detector, a large detector
proposed for {\bf ILC} and {\bf CLIC}. 

\glossterm{INO} India-based Neutrino Observatory, a future underground
laboratory in Tamil Nadu, India.

\glossterm{IOPS}  Input/output Operations Per Second.

\glossterm{IOTA}  Integrable Optics Test Accelerator, an electron
storage  ring to be constructed at {\bf ASTA} to study nonlinear beam optics.

\glossterm{IPPOG} International Particle Physics Outreach Group, a
network of science educators and communicators based at {\bf CERN}.

\glossterm{ISIS} Research center at Rutherford Appleton Laboratory near Oxford.

\glossterm{IsoDAR} Isotope Decay At Rest experiment, a
proposed cyclotron-based sterile neutrino experiment.

\glossterm{ISR} Initial State Radiation, radiation from a high-energy lepton,
quark, or gluon emitted as it scatters from another high-energy
particle.

\glossterm{JLab}  Thomas Jefferson National Accelerator Facility, in
Newport News, Virginia.

\glossterm{J-PARC} Japan Proton Accelerator Research Complex, the laboratory
hosting the major Japanese proton accelerator, located in Tokai.

\glossterm{JEM-EUSO}  Extreme Universe Space Observatory onboard the Japanese
Experiment Module, proposed ultra-high energy cosmic ray detector on
the International Space Station.


\glossterm{JUNO} Jiangmen Underground Neutrino Observatory, a proposed large
scintillator experiment for reactor neutrino oscillations located in China.

\glossterm{K2K} KEK to Kamioka, the first-generation long-baseline
oscillation experiment using beam from {\bf KEK} to {\bf Super-K}.

\glossterm{KamLAND} Kamioka Liquid scintillator ANtineutrino
Detector, a reactor neutrino experiment at Kamioka in Japan.

\glossterm{KamLAND-Zen} Zero neutrino double beta decay search, a 
neutrinolesss double beta decay experiment using a Xe-doped balloon 
deployed in {\bf KamLAND}.
  
\glossterm{KATRIN} KArlsruhe TRItium Neutrino experiment, an experiment to
measure neutrino mass from the endpoint of the tritium beta decay spectrum. 

\glossterm{KEK} Ko-Enerugi Kenkyusho, the major high energy physics laboratory
in Japan, located in Tsukuba.


\glossterm{KLOE, KLOE-2}  An experiment at {\bf LNF} studying
     $\ee$ annihilation at energies near 1~GeV.


\glossterm{KM3NET} Multi-Km$^3$ Neutrino Telescope, a
future deep-sea neutrino telescope in the Mediterranean sea.

\glossterm{KOTO} An experiment at {\bf J-PARC} to study the rare 
decay $K^0_L\to \pi^0 \nu\bar \nu$. 


\glossterm{kt} Kilotonne $= 10^6$~kilograms.

\glossterm{\boldmath$\Lambda$CDM}  The standard model of cosmology in which the Universe
consists of known particles, cold dark matter (CDM), and dark energy
($\Lambda$).

\glossterm{\boldmath$L/E$}   Length ({\bf baseline}) / Energy, a figure of
merit for  neutrino oscillation experiments.

\glossterm{LAGUNA} Large Apparatus studying Grand Unification and Neutrino
Astrophysics, a collaborative project to assess the possibilities for a deep
underground neutrino observatory in Europe; includes the {\bf GLACIER },
{\bf MEMPHYS}, and {\bf LENA} concepts.

\glossterm{LANL} Los Alamos National Laboratory.

\glossterm{LANSCE} Los Alamos Neutron Science Center.

\glossterm{LAPPD} Large Area Picosecond Photo-Detectors project, a
collaboration working to develop large-area flat-panel photon detectors.

\glossterm{LAr} Liquid argon.

\glossterm{LAr1} A proposal to add additional liquid argon TPCs to
the {\bf Fermilab} Booster neutrino beamline.

\glossterm{LAr1-ND} A proposal to add a liquid argon TPC near
detector in the {\bf Fermilab} Booster neutrino beamline.

\glossterm{LArIAT} Liquid Argon In A Testbeam,  
a liquid argon TPC test beam experiment at {\bf Fermilab}.

\glossterm{LARP} LHC Accelerator Research Program, a collaboration
of U.S. national laboratories to develop technology for the current
and future stages of the {\bf LHC}. 

\glossterm{LBNE} Long Baseline Neutrino Experiment:  An experiment to measure
neutrino oscillations, with beams from {\bf Fermilab} and a detector at {\bf SURF}
in South Dakota.

\glossterm{LBNL} Lawrence Berkeley National Laboratory (formerly LBL,
Lawrence Berkeley Laboratory).

\glossterm{LBNO} Long-Baseline Neutrino Oscillation experiment, a 
proposed accelerator-based neutrino oscillation experiment in Europe.

\glossterm{LCF}  Leadership Computing Facility, one of the {\bf DOE}'s 
          high-performance computing centers.

\glossterm{LENA} Low Energy Neutrino Astronomy, a proposed
next-generation liquid scintillator detector.
 
\glossterm{LENS} Low Energy Neutrino Spectroscopy, a low-energy indium-based
solar neutrino experiment.

\glossterm{LEP, LEP-2} The Large Electron-Positron collider at {\bf CERN}, which 
performed precision studies of the $Z$ meson and searches for new
phenomena from 1989 from 2000.

\glossterm{LHC} Large Hadron Collider, a large proton-proton collider
at {\bf CERN}, with design {\bf CM} energy 14 TeV.

\glossterm{LHCb} LHCb, an experiment at the Large Hadron Collider
specialized to measure the production in the forward direction of
hadrons containing heavy quarks.

\glossterm{LHeC} Large Hadron-electron Collider, a proposal for a high-energy
electron beam colliding with the proton beam of the {\bf LHC} at {\bf CERN}. 

\glossterm{LNF}   Laboratori Nazionali di Frascati,  the leading
particle physics laboratory in Italy, located in Frascati, near Rome.


\glossterm{LO} Leading Order, applied to the level of a quantum field
theory calculation (see also {\bf NLO}).

\glossterm{LPA}  Laser Plasma Acceleration.

\glossterm{LSND} Liquid Scintillator Neutrino Detector, an experiment
at Los Alamos National Laboratory searching for sterile neutrinos.

\glossterm{LSO} Lutetium Silicate (Lu$_2$SiO$_5$).

\glossterm{LSST} Large Synoptic Survey Telescope.

\glossterm{LUX} Large Underground Xenon detector, a large dark matter
detector using liquid Xenon.

\glossterm{LVD} Large Volume Detector, a neutrino observatory at {\bf Gran
Sasso} studying low-energy neutrinos from gravitational stellar collapse.

\glossterm{LXe} Liquid Xenon.

\glossterm{LYSO} Lutetium Yttrium OxyorthoSilicate (Lu$_2$(1-x)Y$_2$(x)SiO$_5$).

\glossterm{LZ} LUX-Zeplin, a next-generation xenon TPC dark matter detector.

\glossterm{MAGIC}  Major Atmospheric Gamma-ray Imaging Cherenkov
telescope, a telescope for high-energy gamma rays seen as air showers
in Cherenkov radiation, located in the Canary Islands.

\glossterm{MAJORANA} An experiment searching for neutrinoless double-beta decay
in Ge, located at {\bf SURF}. 

\glossterm{MAMI}  MAinzer MIcrotron, a low-energy electron accelerator
at the University of Mainz.

\glossterm{Many-core}  Refers to computer chips that contain many more
cores than {\bf multi-core CPU}s, i.e., more than  than 16 {\bf CPU}s.

\glossterm{MAPS} Monolithic Active Pixel Sensor.

\glossterm{MCP} Micro-Channel Plate.

\glossterm{MEG}   MuEGamma experiment, an experiment at {\bf PSI} 
 searching for the decay $\mu \to e \gamma$.


\glossterm{MEMPHYS} MEgaton Mass PHYSics, a large water
Cherenkov detector proposed for {\bf CERN} or {\bf ESS}. 

\glossterm{MESA}  Mainz Energy-Recovering Superconducting Accelerator,
a proposed electron accelerator at the Universithy of Mainz.


\glossterm{MicroBooNE} Liquid argon TPC experiment in the Booster neutrino
beamline at {\bf Fermilab}.

\glossterm{MicroMegas} Micro-Mesh gas detector, a type of {\bf MPGD}
ionization detector.

\glossterm{Milagro}  A water Cherenkov gamma-ray telescope located 
  near {\bf LANL}. 



\glossterm{MINER$\nu$A} Main Injector Experiment for $\nu$-A, a
neutrino scattering experiment in the NuMI beamline at {\bf Fermilab}.

\glossterm{MiniBooNE} A short-baseline neutrino oscillation experiment using
a mineral oil-based Cherenkov detector in the Booster neutrino beamline at
{\bf Fermilab}.  

\glossterm{MIND} Magnetised Iron Neutrino Detector, a proposed
neutrino factory detector.

\glossterm{MINOS} Main Injector Neutrino Oscillation Search, a neutrino
oscillation experiment located at {\bf Soudan} using the NuMI beamline at
{\bf Fermilab}.

\glossterm{MKID} Microwave Kinetic Inductance Detector.

\glossterm{MNS} Maki-Nakagawa-Sakata matrix (see {\bf PMNS}).

\glossterm{MOLLER}  An experiment at {\bf JLab} to measure the {\bf
  P}-violating asymmetry in electron-electron scattering.

\glossterm{MPGD} Micro-Pattern Gas Detector.

\glossterm{MPPC} Multi-Pixel Photon Counter.

\glossterm{\boldmath$\overline{MS}$} Modified Minimal Subtraction, a prescription for
removing divergences commonly used in high-precision quantum field theory
calcuations.

\glossterm{MS-DESI} Mid-Scale Dark Energy Spectroscopic Instrument.

\glossterm{MSSM} Minimal Supersymmetric Standard Model, the simplest
(though, not very simple) model that extends the {\bf SM} by the addition of 
{\bf SUSY}. 

\glossterm{MSW} Mikheyev-Smirnov-Wolfenstein effect, the modification of the
neutrino oscillation probability as neutrinos pass through matter.

\glossterm{Multi-core}  Refers to computer chips that contain up to
about 16 {\bf CPU}s.

\glossterm{Muon \boldmath$g-2$}  An experiment at {\bf Fermilab} to 
 measure the anomalous magnetic moment of the muon.

\glossterm{mwe} Meters water-equivalent, a measure of the depth of an
underground detector.

\glossterm{NA62}   An experiment proposed at {\bf CERN} to measure
 the rare kaon decay $K^+ \to \pi^+ \nu \bar \nu$. 

\glossterm{NC} Neutral Current weak interactions.

\glossterm{NERSC}  National Energy Research Scientific Computing Center.

\glossterm{NESSiE} Neutrino Experiment with SpectrometerS in Europe,
a proposed experiment to search for sterile neutrinos using the {\bf CERN}
SPS beam and the {\bf ICARUS} detector.

\glossterm{NEXT} Neutrino Experiment with Xenon TPC, a neutrinoless
double beta decay experiment at the {\bf Canfranc Underground Laboratory}.

\glossterm{NF} Neutrino Factory.

\glossterm{NH} Normal Hierarchy.

\glossterm{NIST} National Institute of Standards and Technology.

\glossterm{NLCTA}  Next Linear Collider Test Accelerator, an
accelerator
at {\bf SLAC} now used for tests of advanced accelerator concepts.

\glossterm{NLO, NNLO} Next-to-Leading Order, Next-to-Next-to-Leading 
Order, terms designating a quantum field theory calculation with two and
three terms, respectively, in the power series in the coupling constant.

\glossterm{NLWCP}  New, Light, Weakly-Coupled Particles.


\glossterm{NNbarX}   An experiment proposed at {\bf Fermilab} to
search
 for neutron-antineutron oscillations.

\glossterm{NOMAD} Neutrino Oscillation MAgnetic Detector,  a
neutrino oscillation experiment at {\bf CERN}.

\glossterm{NO\boldmath$\nu$A} NuMI Off-Axis electron-neutrino Appearance
experiment, a neutrino oscillation experiment in the {\bf NuMI}
beamline at {\bf Fermilab}.

\glossterm{NREN}  National Research and Education Network, a 
     high performance 
   network designed for large scale data movement.

\glossterm{NSF} U.S. National Science Foundation.

\glossterm{NUFO}  National User Facility Organization, the umbrella
group for 
U.S. national user facility users' organizations.

\glossterm{NuMAX} Neutrinos from Muon Accelerators at Project X, a 
proposed neutrino oscillation experiment using a muon-storage ring as
a source of neutrinos.

\glossterm{NuMI} Neutrinos at the Main Injector, a neutrino beamline at
{\bf Fermilab} using the Main Injector, extending to {\bf Soudan} and Ash River, Minnesota.

\glossterm{nuSTORM} Neutrinos from STORed Muons, a proposed
short-baseline neutrino experiment to study sterile neutrinos using
a muon storage ring as a source of neutrinos.

\glossterm{NVRAM}  Non-Volatile Random Access Memory.

\glossterm{OHEP} Office of High Energy Physics of the U.S. {\bf DOE}. 


\glossterm{OLCF}  Oak Ridge Leadership Computing Center.

\glossterm{OPERA} Oscillation Project with Emulsion-tRacking Apparatus, an 
emulsion- and tracker-based neutrino oscillation experiment at {\bf Gran
Sasso}.

\glossterm{ORCA} Oscillation Research with Cosmics in the Abyss,  a
proposed experiment to measure the neutrino mass hierarchy using the
{\bf KM3NeT} neutrino telescope.

\glossterm{ORKA} A proposed {\bf Fermilab} experiment to measure the rate
of the decay $K^+ \to \pi^+\nu\bar \nu$. 

\glossterm{ORNL} Oak Ridge National Laboratory.

\glossterm{OscSNS} Oscillations at the Spallation Neutrino Source, a proposed
sterile neutrino search using the SNS facility.

\glossterm{P} Parity, the inversion of all spatial coordinates.

\glossterm{PAMELA}  Payload for Antimatter Matter Exploration and Light-nuclei
Astrophysics, space-based anti-matter detector.

\glossterm{PANDA}  antiProton ANnihilation at Darmstadt, a
proposed  experiment
at the GSI Helmholtzzentrum in Darmstadt, Germany, studying
proton-antiproton
annihilation at few-GeV energies.

\glossterm{PandaX}  A liquid xenon dark matter experiment to be
located
in the Jin-Ping Underground Laboratory in Sichuan, China.


\glossterm{PB}  PetaByte, equal to $10^{15}$ bytes of information.

\glossterm{PDF} Parton Distribution Function, a function that describes the
internal structure of the proton by giving the momentum distribution of a
particular constituent, for example, the up quark or gluon.

\glossterm{PEN}  An experiment at {\bf PSI} to measure the ratio
of  decay rates  $\pi^+\to e^+\nu$ / $\pi^+ \to \mu^+ \nu$. 


\glossterm{Persistency management}  Management of  persistent data on disk, tape, or other
media. This includes reducing the risk of loss to an appropriate  level.


\glossterm{PEP-II} An $\ee$ collider operated at {\bf SLAC} from 1998
to 2008 in the center-of-mass energy region of 10 GeV. 


\glossterm{PF}   PetaFlop, $10^{15}$ floating point operations
(usually, per second). 


\glossterm{PFA} Particle Flow Analysis, a method for hadron
calorimetry based on separate measurement of the components of a
hadronic shower with charged particles, photons, and neutral hadrons.


\glossterm{PFS}  Prime Focus Spectrograph, wide-field multi-object spectrograph
for the Subaru telescope.

\glossterm{PIENU}  An experiment at {\bf TRIUMF} to measure the ratio
of  decay rates  $\pi^+\to e^+\nu$ / $\pi^+ \to \mu^+ \nu$. 


\glossterm{PINGU}  Precision IceCube Next Generation Upgrade,  a 
proposed low-energy extension to {\bf IceCube}.

\glossterm{Pipeline computing}  Data analysis that proceeds in stages.

\glossterm{pMSSM}  phenomenological Minimal Supersymmetric Standard Model, 
a 19-parameter subspace of the full MSSM.


\glossterm{PMNS} Pontecorvo-Maki-Nakagawa-Sakata matrix, the matrix
linking the mass and flavor eigenstates of neutrinos. 
 
\glossterm{PMT} PhotoMultiplier Tube.

\glossterm{Port}  To adjust and test a computer program to run on a
new architecture, or the result of this process.

\glossterm{PQ}  Peccei-Quinn symmetry, an approximate symmetry of
quark-Higgs boson interactions, which must also be spontaneously
broken, that allows the possible  {\bf CP}-violating term   in {\bf
  QCD}   to be set to zero.  A consequence of this symmetry is the
 existence of the {\bf axion}. 

\glossterm{PREM} Preliminary Reference Earth Model, a model for the
Earth's density distribution.

\glossterm{PREP} Physics Research Equipment Pool.


\glossterm{Project 8} A proposed tritium-based neutrino mass experiment.

\glossterm{Project X} A planned upgrade of the proton accelerator
injector complex at {\bf Fermilab}, with a superconducting proton 
linear accelerator capable of producing multi-megawatt beams.

\glossterm{PROSPECT} Precision Reactor Neutrino Oscillation and Spectrum
Experiment, a U.S.-based reactor short-baseline oscillation search experiment.

\glossterm{PSI} Paul Scherrer Institute, a national laboratory and
accelerator center in Switzerland.

\glossterm{PTOLEMY} Princeton Tritium Observatory for Light Early-universe
Massive neutrino Yield, a proposed relic Big Bang neutrino background
experiment.

\glossterm{PWFA}  Plasma Wake Field Acceleration.


\glossterm{QCD} Quantum Chromodynamics, the well-established theory
describing the strong subnuclear interactions.

\glossterm{QE} Quasi-Elastic scattering.  In neutrino physics, a
reaction 
such as  $\nu n \to \mu^- p$ in which no mesons are produced.  In 
other contexts, a quasi-elastic reaction on a nucleon 
can involve production of a low-lying nucleon resonance.

\glossterm{QED}  Quantum ElectroDynamics.


\glossterm{Qweak}   An experiment
at {\bf JLab}  to 
measure $Q_{weak}$,  the charge with which the proton couples to
the  $Z$ boson at very low momentum transfer.


\glossterm{RADAR} R\&D Argon Detector at Ash River, proposal to add
a {\bf LAr TPC} to the {\bf NO\boldmath$\nu$A} far detector building in Ash River, Minnesota.

\glossterm{RAT}  A simulation and analysis package for optical detectors developed 
   for the Braidwood project and now used within {\bf SNO+}, {\bf
     MiniClean}, and {\bf DEAP}.

\glossterm{REAPR}  Resonantly Enhance Axion Photon Regeneration, an experiment at {\bf Fermilab} to search 
for {\bf axions}  using a laser beam in a strong magnetic field, the
successor to {\bf GammeV-CHASE}.


\glossterm{RENO} Reactor Experiment for Neutrino Oscillations, a reactor
neutrino experiment in South Korea.

\glossterm{RENO-50} A proposed reactor-based experiment with {\bf baseline} $\sim 50$
km to measure the neutrino mass hierarchy with a large scintillator
detector.

\glossterm{RF}  Radio Frequency.

\glossterm{RHIC} Relativistic Heavy Ion Collider, a colliding beam 
acclerator for protons and heavy ions at {\bf BNL}. 



\glossterm{RICE}  Radio Ice Cherenkov Experiment, neutrino detector in Antarctica.

\glossterm{RICOCHET} A proposed bolometric sterile neutrino search
using {\bf CENNS}.

\glossterm{RICH} Ring Imaging CHerenkov detector.

\glossterm{RIKEN} Rikaguka Kenkyujo, a major Japanese research
institute covering physics, chemistry, and engineering, operating
multiple research groups in Japan and one in the U.S. at {\bf BNL}. 

\glossterm{ROI} Region Of Interest.  In neutrino physics,  the region
of a measured energy spectrum where the signal (typically a peak or 
dip from an oscillation) lies.  In collider physics, a region of a
detector
whose data is used in trigger calculations.

\glossterm{RPC} Resistive Plate Chamber, a type of particle tracking detector.

\glossterm{SAGE} Soviet American Gallium Experiment, a solar
neutrino experiment in the {\bf Baksan} Neutrino Observatory in Russia.

\glossterm{SBIR} Small Business Innovation Research, a grant category of the
{\bf DOE} for collaboration of small businesses with experimental projects. 

\glossterm{SciDAC}  Scientific Discovery through Advanced Computation,
     a program of the {\bf DOE}.

\glossterm{Science DMZ}  A portion of a computer network designed for 
   high-performance scientific applications rather than
   general-purpose computing 
such as web browsing.
 

\glossterm{SciNO$\nu$A} A proposed neutrino scattering experiment adding a
fine-grained scintillator detector at the {\bf NO$\nu$A} near detector
site.

\glossterm{SDSS}  Sloan Digital Sky Survey, a survey of more than a quarter of the sky
using the 2.5-meter telescope at Apache Point Observatory, New Mexico.

\glossterm{SiD} Silicon Detector, a detector with silicon tracking 
proposed for {\bf ILC} and {\bf CLIC}. 


\glossterm{SIMD}  Single Instruction Multiple Data style of parallel programming.

\glossterm{SiPM} Silicon PhotoMultiplier, a silicon-based photodetector
operated in Geiger mode.

\glossterm{SKA}  Square Kilometer Array radio telescope.

\glossterm{SLAC}  Stanford Linear Accelerator Center,
a U.S. national laboratory in Menlo Park, California, featuring a 3~km electron linear
accelerator.

\glossterm{SLC} SLAC Linear Collider, an electron-positron collider
operated at {\bf SLAC} from 1989 to 1998 to study the $Z$ boson and 
develop the technology of linear colliders.

\glossterm{SM} Standard Model of particle physics, which describes
the strong, electromagnetic, and weak interactions as mediated by 
vector fields.

\glossterm{SNO} Sudbury Neutrino Observatory, a solar neutrino
experiment located in Sudbury, Ontario, Canada.

\glossterm{SNO+} A successor to the {\bf SNO} experiment aimed at the 
measurement of neutrinoless double-beta decay of tellurium.

\glossterm{SNOLAB} Underground science laboratory in the Vale Creighton Mine
located near Sudbury, Ontario.

\glossterm{SNS} Spallation Neutron Source, at Oak Ridge National Laboratory.

\glossterm{Soudan} An underground laboratory in northern Minnesota,
housing {\bf MINOS} and low-background experiments.

\glossterm{SOX} A chromium and/or cesium source used with the  {\bf
  Borexino}
detector 
to study the reactor neutrino anomaly.

\glossterm{SRF} Superconducting Radio Frequency ({\bf RF}) cavities and 
      associated technology.

\glossterm{STTR} Small business Technology TRansfer, a {\bf DOE}
program for this purpose.

\glossterm{STAR} Solenoidal Tracker at RHIC, a relativistic heavy ion
collider experiment at Brookhaven.

\glossterm{STEM}  Science, Technology, Engineering, and Mathematics, 
   typically describing a domain of education.
 
\glossterm{STEREO} Search for Sterile Neutrinos at ILL reactor, a
reactor short-baseline oscillation search in France.

\glossterm{SPS} Super Proton Synchotron at {\bf CERN}.

\glossterm{Subaru} An optical/infrared telescope on Mauna Kea operated
 by the National Observatory of Japan. 

\glossterm{Super-K} Super-Kamiokande experiment, water Cherenkov detector
in the Kamiokande mine in Japan studying proton decay as well as solar,
atmospheric, and accelerator-produced neutrinos.

\glossterm{SuperKEKB}  A high-luminosity electron-positron collider,
with
     {\bf CM} energy about 10~GeV, at 
      {\bf KEK}.

\glossterm{Super-NEMO} Super Neutrino Ettore Majorana Observatory, a
neutrinoless double beta decay experiment in the Fr\'ejus underground
laboratory in France.

\glossterm{SURF} Sanford Underground Research Facility:  An underground
laboratory in the former Homestake Mine in Lead, South Dakota.

\glossterm{SUSY} SUperSYmmetry, a symmetry that links together {\bf fermions}
and {\bf bosons}.  In any realistic theory, supersymmetry requires a new
space-time structure extending and generalizing that of relativity. 


\glossterm{T} Time reversal, the transformation of reversing the
direction of time.

\glossterm{T2K} Tokai to Kamiokande experiment, a neutrino oscillation
experiment in Japan, using the {\bf J-PARC} neutrino beam and the 
{\bf Super-K} detector.

\glossterm{TA}  Telescope Array, ultra-high-energy cosmic ray detector in Utah.


\glossterm{TCA} Telecommunications Computing Architechture.

\glossterm{TeV} Tera-electron Volt ($10^{12}$ eV), the order-of-magnitude
energy scale of Higgs boson physics.

\glossterm{TDR} Technical Design Report.

\glossterm{TES} Transition Edge Sensor, a cryogenic sensor based on
the very small amount of energy needed to destroy superconductivity in
a thin film painted on the surface of a detector.

\glossterm{TLEP} Triple Large Electron-Positron collider, a proposed $\ee$
collider in a large circular tunnel, with {\bf CM} energies from 90 to 350~GeV.

\glossterm{TPC} Time Projection Chamber, a type of particle detector in which
ionization from a track flows to a wall of a detector and the arrival time
and location are measured, producing a 3-dimensional image of the
track.

\glossterm{TREK}   Time Reversal Experiment with Kaons, an experiment
at {\bf J-PARC} to search for {\bf T}-violating muon polarization in
the 
decay $K^+\to \pi^0 \mu^+ \nu$. 

\glossterm{TRIUMF}  TRI-University Meson Facility, the national accelerator
laboratory of Canada, located in Vancouver.  It is now operated by a consortium
of 15 universities.

\glossterm{UHE} Ultra High Energy, typically applied to cosmic rays with
energies $> 10^{18}$ eV.

\glossterm{VCSEL}  Vertical Cavity Surface Emitting Laser.

\glossterm{VEPP}   One of a  series of electron-positron colliders at the
Budker Institute of Nuclear Physics in Novosibirsk, Russia.

\glossterm{VERITAS} Very Energetic Radiation Imagine Telescope Array 
System,  a telescope for high energy gamma rays seen as air showers
in Cherenkov radiation, located in Arizona.

\glossterm{Vev} Vacuum expectation value, the value of a condensed 
field, such as the Higgs field, at every point in space.

\glossterm{VLHC} Very Large Hadron Collider, a concept for a proton-proton
collider with center of mass energy approximately 100~TeV.

\glossterm{Volunteer computing}  A distributed computing effort in which computer resources
   are donated by the owner, for example, the SETI At Home project.

\glossterm{Wakefield}   The electromagnetic field trailing a bunch of
high-energy particles in an accelerating structure.

\glossterm{WATCHMAN} WATer CHerenkov Monitoring of Anti-Neutrinos, a
collaboration of U.S.-based universities and laboratories conducting
a site search for an advanced water Cherenkov demonstration detector.

\glossterm{WFIRST-AFTA}  Wide-Field InfraRed Survey Telescope-Astrophysics Focused
Telescope Assets

\glossterm{WIMP} Weakly Interacting Massless Particle, a category of
particle that might make up the dark matter of the universe.

\glossterm{WIPP}  Waste Isolation Pilot Plant, an underground facility
located in New Mexico.

\glossterm{YBCO}   Yttrium Barium Copper Oxide,
YBa$_2$Cu$_3$O$_{7-x}$, one of the first high-temperature 
superconductors.